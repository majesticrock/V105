\section{Diskussion}
\label{sec:Diskussion}
Die magnetischen Momente mit relativen Fehlern der drei Messreihen lauten:
\\ \\
\centerline{$\mu_\text{Gravitation} = 0.507 \symup{A m^2} \pm 4.1\% $ }
\centerline{$\mu_\text{Schwingung} =  0.4079 \symup{A m^2} \pm 0.5\% $}
\centerline{$\mu_\text{Präzession} = 0.442  \symup{A m^2} \pm 5.4\%$ .}
\\ \\
Wie die relativen Fehler zeigen, ist die Bestimmungsmethode über die Schwingungsdauer eines Magneten, die die geringste 
Abweichung hat. Da die relativen Fehler bei allen drei Ergebnissen ausschließlich vom Fehler der Steigung der jeweiligen
Ausgleichsgeraden abhängen, ist davon auszugehen, dass die Fehlerwerte somit repräsentativ für die verwendeten, aufgetragenen
Messwerte steht und die Messwerte der Schwingungsdauermessung somit am besten sind. 
Bei der Betrachtung der Abweichung der drei magnetischen Momente untereinander sind folgende Abweichungen festzustellen:
\\ \\
\centerline{$\Delta_\text{$\mu_\text{Gravitation}, \mu_\text{Schwingung}$} = 24 \%$}
\centerline{$\Delta_\text{$\mu_\text{Präzession}, \mu_\text{Schwingung}$} = 8 \%$}
\centerline{$\Delta_\text{$\mu_\text{Gravitation}, \mu_\text{Präzession}$} = 15 \%$.}
\\ \\
Besonders deutlich ist dabei die Abweichung zwischen der Gravitationsmessung und den anderen beiden Messungen. Dies lässt
darauf schließen, dass die Messung mit der Gravitationsmethode merklich schlechter ist als die anderen Messungen.
Diese Abweichungen lassen sich durch bei der Durchführung des Versuches auftretende Fehler erklären.
Bei der Messung des magnetischen Momentes unter Ausnutzung der Gravitation ist die größte Fehlerquelle die analoge Stromstärkenanzeige
des Steuergerätes, welche nur eine Skala von $0.1 \symup{A}$ zulässt und den Wert der Stromstärke so stark verfälschen kann.
Dieser Fehler tritt allerdings bei den beiden weiteren Messungen ebenso auf, weshalb es die starke Abweichung nicht erklären kann.
Des Weiteren ist die Masse, welche am Stiel der Kugel befestigt ist, bei der Messung teilweise etwas verrutscht, da sie sich nicht
gut fixieren ließ, was die Messung des Abstandes zwischen Kugel und Masse beeinträchtigt hat. Ein weiterer Faktor ist die Betrachtung
der angehängten Masse als Punktmasse, sowie die Vernachlässigung der Masse des Aluminiumstabes, wenngleich diese Fehlerquellen eher 
als gering einzuschätzen sind.
Das größte Fehlerpotenzial bei der Messung des magnetischen Momentes über die Schwingungsdauer des Magneten, liegt in der Messung der 
Schwingungsdauer, welche mithilfe einer Stoppuhr per Hand durchgeführt wird und daher sehr fehleranfällig ist.
Der Fehler, der durch die Reibung der Kugel ensteht, ist auf Grund des Luftkissens sehr gering.
Bei der Messung des magnetischen Momentes über die Präzession der Kugel kommt es zu weiteren Fehlern. Das Stroboskop der 
Apparatur ist nicht leicht einzustellen und die eingestellte Frequenz kann sich während der Messung auf Grund der Instabilität des
Stroboskopes verändern.
Hinzu kommt, dass die Präzession der Kugel ebenfalls per Hand angeregt werden muss und daher eine konstante Frequenz nur sehr
schwierig zu ereichen ist, zumal diese im Laufe des Versuches abnimmt. Das heißt, dass längere gemessene Umlaufzeiten etwas stärker 
fehlerbehaftet sind als kürzere. Eine weitere Fehlerquelle ist die erneute Zeitmessung mit der Stoppuhr, welche aber hier
das zusätzliche Problem aufweist, dass ein Start- beziehungsweise Endpunkt der Umdrehung nicht immer ganz konkret auszumachen ist.
Auch leichte Nutationsbewegungen der Kugel sind nicht ganz zu vermeiden, werden aber nur eine untergeordnete Rolle bei der Betrachtung
der Fehler spielen.