\section{Zielsetzung}
In diesem VErsuch soll das magnetische Moment von einem sich in einer Kugel befindlichem Stabmagneten bestimmt werden.
Dazu sollen Methoden mittels der Gravitation, des einer Schwingung und einer Präzessionsbewegung.

\section{Theorie}
\label{sec:Theorie}

Magnetische Feldlinien sind immer in sich selbst geschlossen. Daher kann es keine magnetischen Monopole geben.
Magnetische Dipole können über stromdurchflossene Leiterschleifen oder Permanenmagneten entstehen. Das magnetische Moment von Letzteren ist im Allgemeinen schwierig zu berechnen, kann aber experimentell bestimmt werden.
Dazu wird ein möglichst homogenes Magnetfeld benötigt. Dieses wird in diesem Versuch über Helmholtz-Spulen erreicht.
In der Mitte zwischen den beiden Spulen ist ein Magnetfeld, das als homogen betrachtet werden kann. 
Dieses berechnet sich an dieser Stelle nach

\begin{equation}
    \label{eqn:helmholtz}
    B = \frac{\mu_0 N I R^2}{\big(R^2 + \Big( \frac{d}{2} \Big)^2 \Big)^{\frac{3}{2}}}.
\end{equation}

Dabei ist $\mu_0$ die magnetische Feldkonstante, $R$ der Spulenradius, $N$ die Windungszahl, $I$ die Stromstärke und $d$ der Abstand zwischen den beiden Spulen.


\subsection{Bestimmung mittels Gravitation}

