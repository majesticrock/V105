\section{Zielsetzung}
In diesem Versuch soll das magnetische Moment von einem sich in einer Kugel befindlichem Stabmagneten bestimmt werden.
Dazu sollen Methoden mittels der Gravitation, einer Schwingung und einer Präzessionsbewegung verwendet werden.

\section{Theorie}
\label{sec:Theorie}

Magnetische Feldlinien sind immer in sich selbst geschlossen. Daher kann es keine magnetischen Monopole geben.
Magnetische Dipole können über stromdurchflossene Leiterschleifen oder Permanenmagneten entstehen. Das magnetische Moment von Letzteren ist im Allgemeinen schwierig zu berechnen, kann aber experimentell bestimmt werden.
Dazu wird ein möglichst homogenes Magnetfeld benötigt. Dieses wird in diesem Versuch über Helmholtz-Spulen erzeugt.
In der Mitte zwischen den beiden Spulen ist ein Magnetfeld, das als homogen betrachtet werden kann. 
Dieses berechnet sich an dieser Stelle nach

\begin{equation}
    \label{eqn:helmholtz}
    B = \frac{\mu_0 N I R^2}{\big(R^2 + \Big( \frac{d}{2} \Big)^2 \Big)^{\frac{3}{2}}}.
\end{equation}

Dabei ist $\mu_0$ die magnetische Feldkonstante, $R$ der Spulenradius, $N$ die Windungszahl, $I$ die Stromstärke und $d$ der Abstand zwischen den beiden Spulen.


\subsection{Bestimmung mittels Gravitation}

Eine Masse, die mittels einer stabförmigen Halterung an der Kugel befestigt wird, erfährt durch die Gravitation ein Drehmoment $\mathbf{D}_G$, welches sich mittels der Masse des Objektes $m$, dessen Abstand zum Kugelmittelpunkt $\mathbf{r}$ und dem $\mathbf{g}$-Faktor berechnet. Dabei gilt

\begin{equation}
    \label{eqn:dreh-grav}
    \mathbf{D}_G = m \cdot (\mathbf{r} \times \mathbf{g}).
\end{equation}

Das Magnetfeld lässt auf den Permanenmagneten ebenfalls ein Drehmoment wirken. Dieses berechnet sich nach

\begin{equation}
    \label{eqn:dreh-magn}
    \mathbf{D}_B = \mathbf{µ_\text{Di}} \times \mathbf{B},
\end{equation}

wobei $\mathbf{µ_\text{Di}}$ das magnetische Moment des Permanenmagneten ist.
Da sich $\mathbf{B}$ und $\mathbf{g}$ sich in der Richtung nicht unterscheiden, kann zur Berechnung des Betrages des magnetischen Momentes das Kreuzprodukt durch skalare Produkte ersetzt werden. Die vorkommenden $\sin( \theta )$ Terme kürzen sich jeweils.
Es ergibt sich 

\begin{equation}
    \label{eqn:magn-grav}
    \mu_\text{Di} = \frac{m r g}{B}
\end{equation}

für das magnetische Moment.


\subsection{Bestimmung mittels Schwingung}

Die Schwingung eines Magneten in einem externen Magnetfeld wird mittels der Differentialgleichung des harmonischen Oszillators

\begin{equation}
    \label{eqn:harm-osz}
    -|\mathbf{µ}_\text{Di} \times \mathbf{B}| = J_K \frac{\symup{d}^2 \theta}{\symup{d}t^2}
\end{equation}

beschrieben. Dabei ist $J_K$ das Trägheitsmoment des Objektes, in diesem Versuch der Kugel, und $\theta$ der Auslenkungswinkel.
Das Trägheitsmoment einer Kugel berechnet sich dabei nach

\begin{equation}
    J_K = \frac{2}{5} m_K R_K^2, 
\end{equation}

wobei $m_K$ die Masse und $R_K$ der Radius der Kugel sind.

Mit der Lösung dieser Differentialgleichung lässt sich das magnetische Moment als

\begin{equation}
    \label{eqn:magn-schwing}
    \mu_\text{Di} = \frac{4 \pi^2 J_K}{B T^2}
\end{equation}

beschreiben.


\subsection{Bestimmung mittels Präzessionsbewegung}

Eine Präzessionsbewegung ist jene, die entsteht, wenn die Drehachse von einem rotierenden Körper sich ebenfalls um eine weitere Drehachse dreht. 
Dabei ist letztere parallel zu einem Drehimpuls $L$, der von der Frequenz $f$ und dem oben bereits erwähnten Trägheitsmoment $J_K$ abhängt. Dabei gilt die Relation

\begin{equation}
    \label{eqn:drehimpuls}
    L = 2 \pi f J_K.
\end{equation}

Das Magnetfeld der Helmholtz-Spule wirkt eine äußere Kraft auf die Kugeln, was die Präzessionsbewegung hervorruft. Diese wird mittels der Differentialgleichung

\begin{equation}
    \mathbf{µ}_\text{Di} \times \mathbf{B} = \frac{\symup{d}\mathbf{L}}{\symup{d}t}
\end{equation}

beschrieben. Aus dieser folgt die Präzessionsfrequenz $\Omega_P$ mit

\begin{equation}
    \label{eqn:praez-freq}
    \Omega_P = \frac{\mu_\text{Di}}{L}.
\end{equation}

Daraus folgt direkt

\begin{equation}
    \label{eqn:magn-praez}
    \mu_\text{Di} = \frac{2 \pi L}{B T_P}.
\end{equation}

Dabei ist $T_P$ die Periodendauer der Präzessionsbewegung.