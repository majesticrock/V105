\section{Auswertung}
\label{sec:Auswertung}

%\begin{figure}
%  \centering
%  \includegraphics{plot.pdf}
%  \caption{Plot.}
%  \label{fig:plot}
%\end{figure}

  Die geometrischen Angaben zum Helmholtzspulenpaar sind der Anleitung \cite[2]{V105} entnommen und stimmen mit den am Gerät  
  abzulesenden Daten überein. Die Daten sind in \autoref{tab:spule} zu sehen.
  \begin{table}[!htp]
    \centering
    \caption{Physikalische Angaben zum Helmholtzspulenpaar.}
    \label{tab:spule}
    \begin{tabular}{c c c}
    \toprule
    {Radius der Spulen $R_\text{Spule}$ / m} & {Abstand der beiden Spulen $d$ / m} & {Anzahl Windungen $N$} \\
    \midrule
    0.109 & 0.138 & 195 \\
    \bottomrule
    \end{tabular}
  \end{table}

  Die gemessenen, geometrischen Daten der zu untersuchenden Kugel sind in \autoref{tab:kugel} zu sehen.
  \begin{table}[!htp]
    \centering
    \caption{Geometrische Daten der Kugel.}
    \label{tab:kugel}
    \begin{tabular}{c c c}
    \toprule
    {Radius der Kugel $r_K$ / m} & {Masse der Kugel $m_K$ / kg} & {Länge des Stiels der Kugel  $r_\text{Stiel}$ / m}\\
    \midrule
    0.02615 & 0.14197 & 0.01225 \\
    \bottomrule
    \end{tabular}
  \end{table}
  
  Das Trägheitsmoment der Kugel ist dann somit 
  \\ \\
  \centerline{$J_K = \frac{2}{5} m_K r_K^{2} = 3.88 \cdot 10^{-5} \symup{kg m^2}$.}
  \\ \\

\subsection{Bestimmung des magnetischen Moments mittels Gravitation}
  Die für diesen Versuchsteil benötigte Masse $m_M$, welche in verschieden Abständen zum Stiel der Kugel gemessen wird, hat ein Gewicht von
  $m_M = 1.63 \symup{g}$. 
  Der zur Bestimmung des magnetischen Moments gemessener Zusammenhang zwischen Abstand der Masse $m_M$ und der für das Magnetfeld
  aufgewandte Stromstärke $I$ ist in \autoref{tab:gravi} zu sehen.
  \begin{table}[!htp]
\centering
\caption{Gemessener Zusammenhang zwischen Abstand der Masse $m_M$ zum Stiel an der Kugel und aufgewandter Stromstärke.}
\label{tab:gravi}
\begin{tabular}{c c}
\toprule
{$I$ / A} & {$l_\text{Masse}$ / cm} \\
\midrule
2.5 & 5.460 \\
2.4 & 5.000 \\
2.4 & 4.575 \\
2.2 & 4.045 \\
2.1 & 3.500 \\
1.9 & 3.030 \\
1.9 & 2.505 \\
1.7 & 2.000 \\
1.6 & 1.495 \\
1.5 & 1.000 \\
\bottomrule
\end{tabular}
\end{table}
  Aus den Stromstärken $I$ lassen sich dann mittels Gleichung \eqref{eqn:helmholtz} und den Werten aus \autoref{tab:spule}
   die aufgebrachten Magnetfelstärken errechnen. Die Abstände sind als 
  Abstände zwischen der Masse $M_m$ und dem Zentrum der Kugel zu betrachten, weshalb zu den Abständen aus \autoref{tab:gravi} 
  noch der Radius der Kugel $r_K$ und die Länge des Stieles der Kugel $r_\text{Stiel}$ aus \autoref{tab:kugel} addiert werden müssen.
  Dieser Abstand $r$ wird gegen die magnetische Flussdichte $B$ aufgetragen und anschließend eine lineare Regression der Form 
  \begin{equation}
  \label{eqn:Gerade}
    y = a \cdot x + b
  \end{equation}
  durchgeführt. $y$ beschreibt in diesem Fall die magnetischen Flussdichten $B$
  und $x$ die Abstände $r$, sodass die Steigung a hier durch $a = \frac{B}{r}$ gegeben ist.
  Die Konstanten $a$ und $b$ errechnen sich über 
  \begin{equation}
    \label{eqn:a}
    a = \frac {\sum_{i=1}^N (x_i - \overline{x}) (y_i - \overline{y})}{\sum_{i=1}^N (x_i - \overline{x})^2}
  \end{equation}
  und
  \begin{equation}
  \label{eqn:b}
  b = \overline{y} - a \cdot \overline{x},
  \end{equation}
  wobei die $x_i$ beziehungsweise $y_i$ die einzelnen aufgetragenen Werte darstellen und $\overline{x}$ beziehungsweise $\overline{y}$
  die Mittelwerte dieser nach
  \begin{equation}
  \label{eqn:mittelwert}
  \overline{x} = \frac {1} {N} \sum_{i=1}^N x_i
  \end{equation}
  mit dem Fehler 
  \begin{equation}
  \label{eqn:FehlerMittelwert}
  \Delta \overline{x} = \frac{1}{\sqrt{N \cdot (N-1)}} \sqrt{ \sum_{i=1}^N (x_i - \overline{x})^2}.
  \end{equation}
  Der Fehler von $a$ berechnet sich außerdem jeweils nach Gaußscher Fehlerfortpflanzung zu
  \begin{equation}
  \label{eqn:fehlera}
  \Delta a = \sqrt{\frac{N}{N \sum_{i = 1}^{N} x_i^{2} - (\sum_i^{N} x_i)^2} \cdot \Bigl (\frac {1}{N-2} \sum_{i = 1}^{N}(y_i -b - m x_i )^2 \Bigr) }.
  \end{equation}
  Die Anzahl der verwendeten Messwerte ist in allen Versuchsteilen $N = 10$.
  Die aufgetragenen Werte und die zugehörige Ausgleichsgerade sind in \autoref{fig:gravigraph} zu sehen.
  Die Steigung der Geraden beträgt nach Gleichung \eqref{eqn:a} und \eqref{eqn:fehlera}
  \\ \\
  \centerline{$a = (31.258 \pm 1.324) \symup{\frac{T}{m}}$.}
  \\ \\ 
  Mittels Gleichung \eqref{eqn:magn-grav} lässt sich das magnetische Dipolmoment dann zu
  \\ \\ 
  \centerline{$\mu_\text{Dipol} = (0.507 \pm 0.021) \symup{A m^2}$}
  \\ \\
  bestimmen, wobei sich der Fehler durch Gaußsche Fehlerfortpflanzung errechnet zu 
  \\ \\
  \centerline{$\Delta \mu_\text{Dipol} = m \cdot g \cdot \Delta a$.}
  \\ \\

\begin{table}[!htp]
\centering
\caption{Abstand $r$ zwischen Kugelmittelpunkt und aufgesteckter Masse und zugehörige magnetische Flussdichte $B$.}
\label{tab:gravi2}
\begin{tabular}{c c}
\toprule
{$r$ / m} & {$B$ / $10^{-3}$ T} \\
\midrule
0.093 & 3.390 \\
0.088 & 3.254 \\
0.084 & 3.254 \\
0.079 & 2.983 \\
0.073 & 2.848 \\
0.069 & 2.576 \\
0.063 & 2.576 \\
0.058 & 2.305 \\
0.053 & 2.170 \\
0.048 & 2.034 \\
\bottomrule
\end{tabular}
\end{table}
\begin{figure}
  \centering
  \includegraphics{gravitation.pdf}
  \caption{Auftragung des Abstandes $r$ zwischen Kugelzentrum und augesteckter Masse gegen die magnetische Flussdichte $B$, mit Ausgleichgerade.}
  \label{fig:gravigraph}
\end{figure}

\subsection{Bestimmung des magnetischen Moments über die Schwinungsdauer eines Magneten}
    Die für diesen Auswertungsteil gemessenen Stromstärken $I$ und zehnfache Periodendauern $T$ sind in \autoref{tab:schwing} dargestellt.
    \begin{table}[!htp]
\centering
\caption{Stromstärke $I$ und zugehörige zehnfache Schwingungsdauer $T$.}
\label{tab:schwing}
\begin{tabular}{c c}
\toprule
{$I$ / A} & {$10 \cdot T$ / s} \\
\midrule
0.5 & 23.49 \\
0.9 & 17.60 \\
1.3 & 14.63 \\
1.7 & 12.70 \\
2.1 & 11.61 \\
2.4 & 10.55 \\
2.8 &  9.96 \\
3.2 &  9.27 \\
3.6 &  8.55 \\
4.0 &  8.32 \\
\bottomrule
\end{tabular}
\end{table}
   
    Die quadratischen Periodendauern $T^2$ werden gegen $\frac{1}{B}$ aufgetragen und anschließend mittels linearer Regression eine
    Ausgleichsgerade, wie in Gleichung \eqref{eqn:Gerade}, bestimmt, wobei $y$ in diesem Fall die Kehrwerte der 
    magnetischen Flussdichten $\frac{1}{B}$ darstellt und $x$ die quadratischen Periodendauern $T^2$.
    Die verwendeten Werte sind in \autoref{tab:schwing2} zu sehen; die magnetischen Flussdichten werden wieder mittels Gleichung
    \eqref{eqn:helmholtz} und mithilfe der Werte aus \autoref{tab:spule} bestimmt.
    Die aufgetragenen Werte, sowie die zugehörige Ausgleichgerade finden sich in \autoref{fig:schwingung}.
    Die Steigung der Ausgleichsgerade ist dabei nach Gleichung \eqref{eqn:a} und \eqref{eqn:fehlera}
    \\ \\
    \centerline{$a = T^2 B = (266.07 \pm 1.46) \symup{T s^2} $.}
    \\ \\
    Mittels Gleichung \eqref{eqn:magn-schwing} lässt sich damit das magnetische Moment errechnen zu
    \\ \\
    \centerline{$\mu_\text{Dipol} = (0.4079 \pm 0.0022) \symup{A m^2}$,}
    \\ \\
    wobei sich der zugehörige Fehler über 
    \\ \\
    \centerline{$\Delta \mu_\text{Dipol} = \frac{4 \pi^2 \cdot J_K}{a} \Delta a   $.}
    \\ \\
    Das Trägheitsmoment der Kugel $J_K$ ist bereits zuvor bestimmt worden.
    \begin{table}[!htp]
\centering
\caption{Kehrwert der Magnetfeldstärke $\frac{1}{B}$ und quadratische Periodendauer $T^2$.}
\label{tab:schwing2}
\begin{tabular}{c c}
\toprule
{$\frac{1}{B} / \symup{\frac{1}{T}}$} & {$T^2 / \symup{s^2}$} \\
\midrule
1474.860 & 5.519 \\
819.348 & 3.098 \\
567.241 & 2.140 \\
433.772 & 1.613 \\
351.149 & 1.348 \\
307.255 & 1.113 \\
263.362 & 0.992 \\
230.441 & 0.859 \\
204.837 & 0.731 \\
184.353 & 0.692 \\
\bottomrule
\end{tabular}
\end{table}
    \begin{figure}
        \centering
        \includegraphics{schwingung.pdf}
        \caption{Auftragung der quadratischen Periodendauer $T^2$ gegen den Kehrwert der magnetischen Flussdichte $\frac{1}{B}$, mit Ausgleichgerade.}
        \label{fig:schwingung}
    \end{figure}

    \subsection{Bestimmung des magnetischen Momentes über die Präzession eines Magneten}
        Die zur Bestimmung des magnetischen Momentes gemessenen Stromstärken und die dazugehörigen Umlaufzeiten sind in \autoref{tab:praezess}
        dargestellt. Bei der Messung wird eine Rotationsfrequenz von $f = 4.9 \symup{Hz}$  verwendet, womit sich der Drehimpuls nach
        Gleichung \eqref{eqn:drehimpuls} zu 
        \\ \\
        \centerline{$L_K = 0.0012 \symup{\frac{kg m}{s}}$}
        \\ \\
        errechnet. 
        \begin{table}[!htp]
\centering
\caption{Stromstärke $I$ und zugehörige Periodendauer $T$.}
\label{tab:praezess}
\begin{tabular}{c c}
\toprule
{$I$ / A} & { $T$ / s} \\
\midrule
0.5 & 20.21 \\
0.9 & 11.58 \\
1.3 &  9.80 \\
1.7 &  6.56 \\
2.1 &  6.36 \\
2.4 &  4.73 \\
2.8 &  4.96 \\
3.2 &  3.66 \\
3.6 &  3.55 \\
4.0 &  3.03 \\
\bottomrule
\end{tabular}
\end{table}
        Die Kehrwerte der Periodendauern $\frac{1}{T}$ werden gegen die magnetischen Flussdichten $B$ aufgetragen.
        Die magnetischen Flussdichten $B$ werden wieder durch Gleichung \eqref{eqn:helmholtz} und den Werten aus \autoref{tab:spule}
        bestimmt. Die bei der Auftragung verwendeten Werte sind in \autoref{tab:praezess2} zu sehen. Über die gegeneinander 
        aufgetragenen Werte wird eine lineare Regression, zur Bestimmung einer Ausgleichsgerade derselben Form wie Gleichung \eqref{eqn:Gerade} durchgeführt,
        wobei $y$ hier die magnetischen Flussdichten $B$ und $x$ die Kehrwerte der Periodendauern $\frac{1}{T}$  darstellt.
        Die aufgetragenen Werte, sowie die Ausgleichsgerade sind in \autoref{fig:praezess} dargestellt.
        Die Steigung der Geraden ist nach Gleichung \eqref{eqn:a} und \eqref{eqn:fehlera}
        \\ \\
        \centerline{$a = T \cdot B = 0.0170 \pm 0.0009 \symup{T s}$.}
        \\ \\
        Zusammen mit den bestimmten Drehimpuls $L_K$ lässt sich das magnetische Moment durch Gleichung \eqref{eqn:magn-praez} zu
        \\ \\
        \centerline{$\mu_\text{Dipol} = (0.442 \pm 0.024) \symup{A m^2}$}
        \\ \\
        bestimmen.
        Der Fehler des magnetischen Moments errechnet sich dabei zu 
        \\ \\ 
        \centerline{$\Delta\mu_\text{Dipol} = \frac{2 \pi L}{a} \Delta a $.}
        \\ \\
        \begin{table}[!htp]
\centering
\caption{Magnetische Flussdichte $B$ und Kehrwert der Periodendauer $\frac{1}{T}$.}
\label{tab:praezess2}
\begin{tabular}{c c}
\toprule
{$B / 10^{-3} \cdot \symup{T}$} & {$\frac{1}{T} / \symup{frac{1}{s}}$} \\
\midrule
0.678 & 0.049 \\
1.220 & 0.086 \\
1.763 & 0.102 \\
2.305 & 0.152 \\
2.848 & 0.157 \\
3.255 & 0.211 \\
3.797 & 0.202 \\
4.339 & 0.273 \\
4.882 & 0.282 \\
5.424 & 0.330 \\
\bottomrule
\end{tabular}
\end{table}
          \begin{figure}
        \centering
        \includegraphics{praezess.pdf}
        \caption{Auftragung des Kehrwertes der Periodendauer $\frac{1}{T}$ gegen die magnetische Flussdichte $B$, mit Ausgleichgerade.}
        \label{fig:praezess}
    \end{figure}