\section{Durchführung}
\label{sec:Durchführung}

Eine Kugel, in der ein sich ein Permanentmagnet befindet, wird auf einen Zylinder in der Mitte von zwei Helmholtz-Spulen gelegt.
Über ein Steuerelement lässt sich ein Luftkissen zwischen Kugel und Zylinder einstellen, sowie die Stromstärke, die durch die Spulen fließt und ein Stroboskop. 
Letzteres ist am Rand der oberen Spule verbaut.
Zunächst werden die Masse und der Radius der Kugel und der Radius und der Abstand der Spulen abgemessen.

In dem gesamten Versuch wird der Spulenstrom zwischen den Messungen abgeschaltet, da die Spulen sich sonst stark erwärmen, was deren elektrischen Widerstand erhöht und die Messung verfälscht.

\subsection{Bestimmung mittels Gravitation}

In dieser Messreihe wird eine verschiebbare Masse an einem Aluminiumstab angebracht. Die Masse wird im Weiteren als Punktmasse angenommen und die Mase des Aluminiumstabes wird vernachlässigt.
Die Masse wird zunächst gewogen.
Dieses System wird in die Kugel eingelassen und es wird die Distanz von der Masse zu der Oberfläche der Kugel gemessen. Zusammen mit dem Radius der Kugel ergibt dies den Abstand von deren Zentrum.
Nun wird das Luftkissen eingeschaltet und zunächst die Stromstärke maximal eingestellt. Letztere wird nun langsam so gesenkt, dass die Kugel kurz vor dem Kippen ist. Der zugehörige Wert wird notiert.
Diese Messung wird insgesamt zehn mal mit verschiedenen Abständen von der Masse zu der Kugel durchgeführt.

\subsection{Bestimmung mittels Schwingung}

Das Luftkissen und das Magnetfeld wird eingeschaltet. Die Kugel wird aus ihrer Ruhelage leicht ausgelenkt und somit in eine Schwingungsbewegung versetzt.
Nun wird die Zeit gemessen, die das System benötigt um zehn Perioden zu absolvieren, um so einen Wert für die Periodendauer zu errechnen.
Diese Messung wird erneut für zehn verschiedene Stromstärken durchgeführt.

\subsection{Bestimmung mittels Präzessionsbewegung}

Die Kugel wird händisch in eine Rotationsbewegung versetzt. Dabei soll die Drehachse konstant senkrecht nach oben zeigen.
Anschließend wird das Stroboskop eingeschaltet und so in der Frequenz geregelt, dass ein weißer Punkt auf der Kugel als sich nicht bewegend wahrgenommen wird. Die Frequenz des Stroboskops ist dann die Drehfrequenz der Kugel.
Es wird darauf geachtet, dass diese Frequenz zwischen 4 und 6 Hz liegt.
Nun wird die Kugel leicht ausgelenkt und das Magnetfeld aktiviert, um so die Präzessionsbewegung hervorzurufen.
Es wird nun die Periodendauer dieser Bewegung bestimmt.
Diese Messung wird ebenfalls für zehn verschiedene Stromstärken durchgeführt.